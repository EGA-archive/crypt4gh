%%%%%%%%%%%%%%%%%%%%%%%%%%%%%%%%%%%%%%%%%%%%%%%%%%%%%%%%%%%%%
\section{Header}\label{header}
%%%%%%%%%%%%%%%%%%%%%%%%%%%%%%%%%%%%%%%%%%%%%%%%%%%%%%%%%%%%%

%%%%%%%%%%%%%%%%%%%%%%%%%%%%%%%%%%%%%%%%%%%%%%%%%%%%%%%%%%%%%
\subsection{Preamble}\label{header:preamble}
%%%%%%%%%%%%%%%%%%%%%%%%%%%%%%%%%%%%%%%%%%%%%%%%%%%%%%%%%%%%%

The file starts with a header, with the following structure:

\begin{verbatim}
enum EncryptionMethod<le_uint32> {
  chacha20_ietf_poly1305 = 0;
};

struct Header {
  byte                             magic_number[8];
  le_uint32                        version;
  le_uint32                        header_len;
  enum EncryptionMethod<le_uint32> header_encryption_method;
  select (header_encryption_method) {
    case chacha20_ietf_poly1305:
      byte                         public_key[32];
      byte                         nonce[12];
  };
  byte[]                           encrypted_header;
};
\end{verbatim}

[magic number][version][header len][header method=0][public key][nonce][encrypted header][mac]

The \kw{magic\_number} is the ASCII representation of the string ``crypt4gh''.

The version number is stored as a four-byte little-endian unsigned integer.
%
The current version number is 1.

\kw{header\_len} is the length of the \emph{remainder} of the header, stored as a four-byte little-endian unsigned integer.
%

\kw{header\_encryption\_method} is an enumerated type that describes the algorithm used to encrypt the parameters for the encryption of the data portion.
%
In the case of \kw{chacha20\_ietf\_poly1305}, these include the 32-byte public key of the encrypter, the 12-byte nonce (See \ref{data:encrypted:chacha20poly1305}).

The current byte representation of the magic number and version is:
\begin{verbatim}
0x63 0x72 0x79 0x70 0x74 0x34 0x67 0x68 0x01 0x00 0x00 0x00
============= magic_number============= ===== version =====
\end{verbatim}



%%%%%%%%%%%%%%%%%%%%%%%%%%%%%%%%%%%%%%%%%%%%%%%%%%%%%%%%%%%%%
\subsection{Header Data}\label{header:data}
%%%%%%%%%%%%%%%%%%%%%%%%%%%%%%%%%%%%%%%%%%%%%%%%%%%%%%%%%%%%%

The parameters used for the encryption of the data portion are encoded using the type \kw{EncryptionParameters}, as follows:

\begin{verbatim}
enum ChecksumAlgorithm<le_uint32> {
  none = 0;
  md5 = 1;
  sha256 = 2;
};

struct EncryptionParameters {
  enum ChecksumAlgorithm<le_uint32> checksum_algorithm;

  enum EncryptionMethod<le_uint32> method;
  select (method) {
    case chacha20_ietf_poly1305:
      byte       key[32];
  };
};
\end{verbatim}

% ---------------
\kw{method} is an enumerated type that describes the type of encryption to be used.

\kw{key} is a secret encryption key.
%
In the case of \kw{chacha20\_ietf\_poly1305}, it is treated as a concatenation of eight 32-bit little-endian integers.

% ---------------
\kw{checksum\_algorithm} is an enumerated type that describes the algorithm used for the checksum to be used for the \emph{unencryted} data content.
%
If a checksum algorithm is chosen, `sha256' SHOULD be prefered and `md5' SHOULD only be used for backwards compatibility.
%
Moreover, the checksum value is of the following form and appended at the end of the encrypted data portion (see Section~\ref{encrypted:data}). Nothing is appended if \kw{checksum\_algorithm} is \kw{none}.
%
\begin{verbatim}
select (checksum_algorithm) {
  case md5:
    byte       key[16];
  case sha256:
    byte       key[32];
  };
\end{verbatim}

%%%%%%%%%%%%%%%%%%%%%%%%%%%%%%%%%%%%%%%%%%%%%%%%%%%%%%%%%%%%%
\subsection{Encrypted Header}\label{header:encrypted}
%%%%%%%%%%%%%%%%%%%%%%%%%%%%%%%%%%%%%%%%%%%%%%%%%%%%%%%%%%%%%

The header data is encrypted using Curve25519-based asymmetric encryption.
%
Informally, Curve25519-based asymmetric encryption uses the Currve25519 ECC function to generate a shared encryption key from the encrypter's private key and the recipient's public key, which can be re-created by the recipient using its secret key and the encrypter's public key.
%

The header data is then encrypted 
%

% ---------------------------------
Finally, the \kw{encrypted\_header} is generated by encrypting the header data, following the \kw{header\_encryption\_method}.
%
In the case of \kw{chacha20\_ietf\_poly1305}, it is encrypted using ChaCha20 and authenticated using Poly1305~\cite{RFC8439}, using the same algorithms described in Section~\ref{data:encrypted:chacha20poly1305}.
%
The public key of the encrypter and the randomly-generated nonce, used in the encryption, are prepended, and the Poly1305 authentication tag is appended.


The shared key, used by Chacha20, is calculated using X25519 ECC function as it is described in~\cite{RFC7748} (section 5).
%

The public key can be used as proof of origin of the encrypted file.
%


The following is an example of a header configuration:
%
\begin{verbatim}
[magic number][version][header len][signing key][nonce][encrypted header]
\end{verbatim}

